\documentclass[11pt,fleqn]{article}

\setlength {\topmargin} {-.15in}
\setlength {\textheight} {8.6in}

\usepackage{amsmath}
\usepackage{amssymb}
\usepackage{amsthm}
\usepackage{color}
\usepackage{fullpage}

\renewcommand{\labelenumi}{\theenumi.}
\renewcommand{\labelenumii}{\theenumii.}
\renewcommand{\labelenumiii}{\theenumiii.}
\newcommand{\be}{\begin{enumerate}}
\newcommand{\ee}{\end{enumerate}}
\newcommand{\bi}{\begin{itemize}}
\newcommand{\ei}{\end{itemize}}
\newcommand{\bc}{\begin{center}}
\newcommand{\ec}{\end{center}}
\newcommand{\bsp}{\begin{sloppypar}}
\newcommand{\esp}{\end{sloppypar}}
\newcommand{\mname}[1]{\mbox{\sf #1}}
\newcommand{\pnote}[1]{{\langle \text{#1} \rangle}}

\begin{document}

\begin{center}

  {\large \textbf{COMPSCI/SFWRENG 2FA3}}\\[2mm]
  {\large \textbf{Discrete Mathematics with Applications II}}\\[2mm]
  {\large \textbf{Winter 2020}}\\[8mm]
  {\huge \textbf{Assignment 2}}\\[6mm]
  {\large \textbf{Dr.~William M. Farmer}}\\[2mm]
  {\large \textbf{McMaster University}}\\[6mm]
  {\large Revised: January 25, 2020}

\end{center}

\medskip

Assignment 2 consists of two problems.  You must write your solutions
to the problems using LaTeX.

Please submit Assignment~2 as two files,
\texttt{Assignment\_2\_\emph{YourMacID}.tex} and
\texttt{Assignment\_2\_\emph{YourMacID}.pdf}, to the Assignment~2
folder on Avenue under Assessments/Assignments.
\texttt{\emph{YourMacID}} must be your personal MacID (written without
capitalization).  The \texttt{Assignment\_2\_\emph{YourMacID}.tex}
file is a copy of the LaTeX source file for this assignment
(\texttt{Assignment\_2.tex} found on Avenue under
Contents/Assignments) with your solution entered after each problem.
The \texttt{Assignment\_2\_\emph{YourMacID}.pdf} is the PDF output
produced by executing

\begin{itemize}

  \item[] \texttt{pdflatex Assignment\_2\_\emph{YourMacID}}

\end{itemize}

This assignment is due \textbf{Sunday, February 2, 2020 before
  midnight.}  You are allow to submit the assignment multiple times,
but only the last submission will be marked.  \textbf{Late submissions
  and files that are not named exactly as specified above will not be
  accepted!}  It is suggested that you submit your preliminary
\texttt{Assignment\_2\_\emph{YourMacID}.tex} and
\texttt{Assignment\_2\_\emph{YourMacID}.pdf} files well before the
deadline so that your mark is not zero if, e.g., your computer fails
at 11:50 PM on February 2.

\textbf{Although you are allowed to receive help from the
  instructional staff and other students, your submission must be your
  own work.  Copying will be treated as academic dishonesty! If any of
  the ideas used in your submission were obtained from other students
  or sources outside of the lectures and tutorials, you must
  acknowledge where or from whom these ideas were obtained.}

\newpage

\subsection*{Problems}

\be

  \item \textbf{[10 points]} Let $C$ be a set of \emph{coefficients}
    closed under addition and multiplication such as the integers
    $\mathbb{Z}$, the rational numbers $\mathbb{Q}$, or the real
    numbers $\mathbb{R}$.  A \emph{polynomial over $C$} is a
    mathematical expression that is constructed from an
    \emph{indeterminant} $x$ and members of $C$ by applying addition
    ($+$) and multiplication ($*$) operators.  For example, $x * ((2 *
    x) + 3)$ is a polynomial over $\mathbb{Z}$.  Let $P$ be the set of
    polynomials over some $C$.  The \emph{value of a polynomial $p \in
      P$ at $c \in C$} is the result of replacing the indeterminant
    $x$ with $c$.  For example, the value of $x * ((2 * x) + 3)$ at 5
    is $5 * ((2 * 5) + 3) = 65$.

Let \mname{Poly} be the inductive type defined by the following
constructors:

  \bi

    \item[] $\mname{X} : \mname{Poly}$.

    \item[] $\mname{Coeff} : \mathbb{Q} \rightarrow \mname{Poly}$.

    \item[] $\mname{Sum} : \mname{Poly} \times \mname{Poly} \rightarrow \mname{Poly}$.

    \item[] $\mname{Prod} : \mname{Poly} \times \mname{Poly} \rightarrow \mname{Poly}$.

  \ei

  The members of \mname{Poly} represent the polynomials over
  $\mathbb{Q}$.  Define \[\mname{val} : \mname{Poly} \times \mathbb{Q}
  \rightarrow \mathbb{Q}\] by structural recursion using pattern
  matching so that, for all $p \in \mname{Poly}$ and $q \in
  \mathbb{Q}$, $\mname{val}(p,q)$ is the value of $p$ at $q$.

  \bigskip

  \textcolor{blue}{\textbf{Name: Hishmat Salehi \\ MacID: salehh6 \\ Date: \today}}

\Large \mname{val(X, q) $=$ q}\\
\mname{val(Coeff(a), q) $= a$}, \textbf{where a $\in \mathbb{Q}$}\\
\mname{val(Sum(a, b), q) $=$ val(a, q) + val(b, q)}, \textbf{where a, b $\in$ Poly}\\
\mname{val(Prod(a, b), q) $=$ val(a, q) * val(b, q)}, \textbf{where a, b $\in$ Poly}\\

\normalsize

  \bigskip

  \item \textbf{[10 points]} Let \mname{Bit} be the inductive set
    defined by the following constructors:

  \bi

    \item[] $\mname{Zero} : \mname{Bit}$.

    \item[] $\mname{One} : \mname{Bit}$.

  \ei

  The members of \mname{Bit} represent 0 and 1.

  Let \mname{BinNum} be the inductive set defined by the following
  constructors:

  \bi

    \item[] $\mname{Nil} : \mname{BinNum}$.

    \item[] $\mname{Join} : \mname{BinNum} \times \mname{Bit} \rightarrow \mname{BinNum}$.

  \ei

  The members of \mname{BinNum} not equal to \mname{Nil} represent
  binary numerals; \mname{Nil} represents an empty numeral.  For
  example, \[\mname{Join}(\mname{Join}(\mname{Join}(\mname{Nil},
  \mname{One}), \mname{Zero}), \mname{One}),\] represents the binary
  number 101.

  The function \[\mname{len} : \mname{BinNum} \rightarrow \mathbb{N}\]
  maps a member of \mname{BinNum} to its length.  \mname{len} is
  defined by the following equations using recursion and pattern
  matching:

  \bi

    \item[] $\mname{len}(\mname{Nil}) = 0$.

    \item[] $\mname{len}(\mname{Join}(u,b)) = \mname{len}(u) + 1$.

  \ei

  The function \[\mname{val} : \mname{BinNum} \rightarrow \mathbb{N}\]
  maps a member of \mname{BinNum} to the value of the binary numeral
  it represents.  For
  example, \[\mname{val}(\mname{Join}(\mname{Join}(\mname{Join}(\mname{Nil},
  \mname{One}), \mname{Zero}), \mname{One})) = (101)_2 = 5.\] \mname{val} is
  defined by the following equations using recursion and pattern
  matching:

  \bi

    \item[] $\mname{val}(\mname{Nil}) = 0$.

    \item[] $\mname{val}(\mname{Join}(u,\mname{Zero})) = 2 * \mname{val}(u)$.

    \item[] $\mname{val}(\mname{Join}(u,\mname{One})) = (2 * \mname{val}(u)) +  1$.

  \ei

  The function \[\mname{add} : \mname{BinNum} \times \mname{BinNum}
  \rightarrow \mname{BinNum}\] is intended to implement addition on
  members of \mname{BinNum}.  It is defined by the following equations
  using recursion and pattern matching:

  \bi

    \item[] $\mname{add}(u,\mname{Nil}) = u$.

    \item[] $\mname{add}(\mname{Nil},u) = u$.

    \item[] $\mname{add}(\mname{Join}(u,\mname{Zero}), \mname{Join}(v,
      \mname{Zero})) = \mname{Join}(\mname{add}(u,v),\mname{Zero}).$

   \item[] $\mname{add}(\mname{Join}(u,\mname{One}), \mname{Join}(v,
      \mname{Zero})) = \mname{Join}(\mname{add}(u,v),\mname{One}).$

    \item[] $\mname{add}(\mname{Join}(u,\mname{Zero}), \mname{Join}(v,
      \mname{One})) = \mname{Join}(\mname{add}(u,v),\mname{One})$.

    \item[] $\mname{add}(\mname{Join}(u,\mname{One}), \mname{Join}(v,
      \mname{One})) = {}$\\ 
      \hspace*{2ex}$\mname{Join}(\mname{add}(\mname{add}(u,v),
      \mname{Join}(\mname{Nil}, \mname{One})),\mname{Zero})$.

  \ei

  Notice that the algorithm behind the definition is essentially the
  same algorithm that children learn to add numbers represented as
  decimal numerals.  The last equation is a bit complicated because it
  involves a carry of 1.

  \textbf{Lemma 1}. For all $u,v \in \mname{BinNum}$,
  \[\mname{len}(\mname{add}(u,v)) \le \mname{len}(u) +
  \mname{len}(v).\]

  \textbf{Theorem 1}. For all $u,v \in
  \mname{BinNum}$, \[\mname{val}(\mname{add}(u,v)) = \mname{val}(u) +
  \mname{val}(v).\]

  Theorem 1 states that \mname{add} correctly implements addition on
  the members of $\mname{BinNum}$.

  Prove Theorem 1 assuming Lemma 1.  (You are not required to prove
  Lemma 1.)  Hint: Use strong induction with $P(n) \equiv
  \mname{val}(\mname{add}(u,v)) = \mname{val}(u) + \mname{val}(v)$ for
  all $u,v \in \mname{BinNum}$ such that $n = \mname{len}(u) +
  \mname{len}(v)$.

  \bigskip

  \textcolor{blue}{\textbf{Name: Hishmat Salehi \\ MacID: salehh6 \\ Date: \today}}

  \textcolor{blue}{\textbf{Problem 2}}
\begin{proof}
Assuming Lemma 1, let $P(n) \equiv \mname{val(add(u, v))} = \mname{val(u) + val(v)}$  for all $u, v \in$ \mname{BinNum} such that n $=$ \mname{len(u) + len(v)}.  We will prove $P(n)$ for all $u, v \in$ \mname{BinNum} by strong induction.

\medskip

\textcolor{red}{Base case}: $n = 0$. We must show $P(0)$. Since $n = 0$, then $0 = \mname{len(Nil) + len(Nil)}$ we have u = Nil and v = Nil.
\begin{align*}
  &\phantom{{}=}  \mname{val(add(Nil, Nil))} & \pnote{LHS of $P(0)$}\\
  &=  \mname{val(Nil)}          & \pnote{definition of \mname{add(a, b)} when $b = Nil$}\\
  &=  0        & \pnote{definition of \mname{val(a)} when $a = Nil$}\\
  &=  0 + 0        & \pnote{arithmetic; definition of \mname{val(a)} when $a = Nil$}\\
  &=  \mname{val(Nil) + val(Nil)}    & \pnote{arithmetic; RHS of $P(0)$}
\end{align*}
So $P(0)$ holds.

\medskip

\textcolor{red}{Base case}: $n = 1$.  We must show $P(1)$. Since $n = 1$, then $1 = \mname{len(Join(Nil, Bit)) + len(Nil)}$ we have u = Join(Nil, Bit) and v = Nil.\\
\begin{align*}
  &\phantom{{}=} \mname{val(add(Join(Nil, Bit), Nil))} & \pnote{LHS of $P(1)$}\\
  &= \mname{val(Join(Nil, Bit))}         & \pnote{definition of \mname{add(a, b)} when $b = Nil$}\\
  &= \mname{val(Join(Nil, Bit))} + 0         & \pnote{arithmetic}\\
  &= \mname{val(Join(Nil, Bit)) + val(Nil)}         & \pnote{definition of \mname{val(b)}, where \mname{b = Nil}; RHS of $P(1)$}\\
\end{align*}
The same proof can be used to prove P(1) when v = Join(Nil, Bit) and u = Nil, \textbf{without loss of generality}.
So $P(1)$ holds.

\medskip

\textcolor{red}{Induction step}: $n > 1$. Assume $P(2), P(3), ... , P(n-2), P(n-1)$. We must show $P(n)$.

  \bi

    \item[] $\mname{val(add(u,Nil)) = val(u)}$.

    \item[] $\mname{val(add(Nil,u)) = val(u)}$.

    \item[] $\mname{val(add(Join(u,Zero), Join(v,Zero))) = val(Join(add(u,v),Zero))}.$

   \item[] $\mname{val(add(Join(u,One), Join(v,Zero))) = val(Join(add(u,v),One))}.$

    \item[] $\mname{val(add(Join(u,Zero), Join(v,One))) = val(Join(add(u,v),One))}.$

    \item[] $\mname{val(add(Join(u,One), Join(v,One))) = val(Join(add(add(u,v), Join(Nil, One)),Zero))}$.

  \ei

With pattern matching recursively and using Inductive hypothesis  $P(0), P(1), ... , P(n-2), P(n-1)$ we know that \mname{len(add(u, v)) $\le$ len(u) + len(v) $=$ n}, then P(n) takes 1 more step than P(n-1) therefore it is finite and $P(n)$ holds.

\medskip

Therefore, $P(n)$ holds for all $n \in \mathbb{N}$ by strong induction.
\end{proof}

\ee

\end{document}


