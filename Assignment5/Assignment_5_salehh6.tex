\documentclass[11pt,fleqn]{article}

\setlength {\topmargin} {-.15in}
\setlength {\textheight} {8.6in}

\usepackage{amsmath}
\usepackage{amssymb}
\usepackage{color}
\usepackage{tikz}
\usepackage{fullpage}
\usetikzlibrary{automata,positioning,arrows}

\renewcommand{\labelenumi}{\theenumi.}
\renewcommand{\labelenumii}{\theenumii.}
\renewcommand{\labelenumiii}{\theenumiii.}
\newcommand{\be}{\begin{enumerate}}
\newcommand{\ee}{\end{enumerate}}
\newcommand{\bi}{\begin{itemize}}
\newcommand{\ei}{\end{itemize}}
\newcommand{\bc}{\begin{center}}
\newcommand{\ec}{\end{center}}
\newcommand{\bsp}{\begin{sloppypar}}
\newcommand{\esp}{\end{sloppypar}}
\newcommand{\mname}[1]{\mbox{\sf #1}}
\newcommand{\sB}{\mbox{$\cal B$}}
\newcommand{\sC}{\mbox{$\cal C$}}
\newcommand{\sF}{\mbox{$\cal F$}}
\newcommand{\sM}{\mbox{$\cal M$}}
\newcommand{\sP}{\mbox{$\cal P$}}
\newcommand{\sV}{\mbox{$\cal V$}}
\newcommand{\set}[1]{{\{ #1 \}}}
\newcommand{\Neg}{\neg} 
\ifdefined \And 
\renewcommand{\And}{\wedge}
\else
\newcommand{\And}{\wedge}
\fi
\newcommand{\Or}{\vee}
\newcommand{\Implies}{\Rightarrow}
\newcommand{\Iff}{\Leftrightarrow}
\newcommand{\Forall}{\forall}
\newcommand{\ForallApp}{\forall\,}
\newcommand{\Forsome}{\exists}
\newcommand{\ForsomeApp}{\exists\,}
\newcommand{\mdot}{\mathrel.}

\begin{document}

\begin{center}

  {\large \textbf{COMPSCI/SFWRENG 2FA3}}\\[2mm]
  {\large \textbf{Discrete Mathematics with Applications II}}\\[2mm]
  {\large \textbf{Winter 2020}}\\[8mm]
  {\huge \textbf{Assignment 5}}\\[6mm]
  {\large \textbf{Dr.~William M. Farmer}}\\[2mm]
  {\large \textbf{McMaster University}}\\[6mm]
  {\large Revised: February 14, 2020}

\end{center}

\medskip

Assignment 5 consists of two problems.  You must write your solutions
to the problems using LaTeX.

Please submit Assignment~5 as two files,
\texttt{Assignment\_5\_\emph{YourMacID}.tex} and
\texttt{Assignment\_5\_\emph{YourMacID}.pdf}, to the Assignment~5
folder on Avenue under Assessments/Assignments.
\texttt{\emph{YourMacID}} must be your personal MacID (written without
capitalization).  The \texttt{Assignment\_5\_\emph{YourMacID}.tex}
file is a copy of the LaTeX source file for this assignment
(\texttt{Assignment\_5.tex} found on Avenue under
Contents/Assignments) with your solution entered after each problem.
The \texttt{Assignment\_5\_\emph{YourMacID}.pdf} is the PDF output
produced by executing

\begin{itemize}

  \item[] \texttt{pdflatex Assignment\_5\_\emph{YourMacID}}

\end{itemize}

This assignment is due \textbf{Sunday, March 1, 2020 before
  midnight.}  You are allow to submit the assignment multiple times,
but only the last submission will be marked.  \textbf{Late submissions
  and files that are not named exactly as specified above will not be
  accepted!}  It is suggested that you submit your preliminary
\texttt{Assignment\_5\_\emph{YourMacID}.tex} and
\texttt{Assignment\_5\_\emph{YourMacID}.pdf} files well before the
deadline so that your mark is not zero if, e.g., your computer fails
at 11:50 PM on March 1.

\textbf{Although you are allowed to receive help from the
  instructional staff and other students, your submission must be your
  own work.  Copying will be treated as academic dishonesty! If any of
  the ideas used in your submission were obtained from other students
  or sources outside of the lectures and tutorials, you must
  acknowledge where or from whom these ideas were obtained.}

\newpage

\subsection*{Presenting DFAs and NFAs Transition Diagrams}

In this assignment you are asked to present DFAs as transition
diagrams.  You are can do this in one of two ways.

The first way is to present the diagram using the LaTeX graphics
package TikZ.  The TikZ code can either be written by hand or
automatically generated using the finsm system available at
\texttt{http:finsm.io}.

Here are some examples of how it can be used to create
DFA and NFA transition diagrams that appear in the lectures slides:

\begin{center}
\begin{tikzpicture}[shorten >=1pt,node distance=2.5cm,on grid,auto] 
   \node[state, initial, thick] (q_0)   {$q_0$}; 
   \node[state, thick] (q_1) [right=of q_0] {$q_1$}; 
   \node[state, thick] (q_2) [right=of q_1] {$q_2$}; 
   \node[state, accepting, thick] (q_3) [right=of q_2] {$q_3$};
    \path[->, thick, >=stealth] 
    (q_0) edge [bend left, above] node {$a$} (q_1)
          edge [loop, above] node {$b$} (q_2)
    (q_1) edge [bend left, above] node {$a$} (q_2)
          edge [bend left, above] node {$b$} (q_0)
    (q_2) edge [bend left, above] node {$a$} (q_3) 
          edge [bend left, below]  node {$b$} (q_0)
    (q_3) edge [loop, above] node {$a,b$} (q_2); 
\end{tikzpicture}
\end{center}

\begin{center}
\begin{tikzpicture}[shorten >=1pt,node distance=4cm,on grid,auto] 
   \node[state, initial, accepting, thick] (q_0)   {$q_0$}; 
   \node[state, thick] (q_1) [right=of q_0] {$q_1$}; 
   \node[state, thick] (q_2) [below=of q_0] {$q_2$}; 
   \node[state, thick] (q_3) [right=of q_2] {$q_3$};
    \path[->, thick, >=stealth] 
    (q_0) edge [bend left, above] node {1} (q_1)
          edge [bend left, right] node {0} (q_2)
    (q_1) edge [bend left, below] node {1} (q_0)
          edge [bend left, right] node {0} (q_3)
    (q_2) edge [bend left, above] node {1} (q_3) 
          edge [bend left, left]  node {0} (q_0)
    (q_3) edge [bend left, below] node {1} (q_2) 
          edge [bend left, left]  node {0} (q_1);
\end{tikzpicture}
\end{center}

\begin{center}
\begin{tikzpicture}[shorten >=1pt,node distance=1.7cm,on grid,auto] 
   \node[state, initial, thick] (q_0)   {$q_0$}; 
   \node[state, thick] (q_1) [right=of q_0] {$q_1$}; 
   \node[state, thick] (q_2) [right=of q_1] {$q_2$}; 
   \node[state, thick] (q_3) [right=of q_2] {$q_3$};
   \node[state, thick] (q_4) [right=of q_3] {$q_4$};
   \node[state, thick, accepting] (q_5) [right=of q_4] {$q_5$};
    \path[->, thick, >=stealth] 
    (q_0) edge [loop, above] node {0,1} (q_1)
          edge [right, above] node {1} (q_1)
    (q_1) edge [right, above] node {0,1} (q_2)
    (q_2) edge [right, above] node {0,1} (q_3)
    (q_3) edge [right, above] node {0,1} (q_4)
    (q_4) edge [right, above] node {0,1} (q_5);
\end{tikzpicture}
\end{center}

The second way is to take a picture of a hand-written transition
diagram and then embed it into your assignment using the following
LaTeX code:
\begin{verbatim}
\begin{center}
\includegraphics[scale = 0.5]{diagram.jpg}
\end{center}
\end{verbatim}
Please make sure your diagram is legible.

\subsection*{Problems}

\be

  \item \textbf{[10 points]} Construct a deterministic finite
    automaton $M$ for the alphabet $\Sigma = \set{a}$ such that $L(M)$
    is the set of all strings in $\Sigma^*$ whose length is divisible by
    either 2 or 5.  Present $M$ as a transition diagram.

  \bigskip

  \textcolor{blue}{\textbf{Name: Hishmat Salehi\\MacID: salehh6\\Date: \today}}

  \textcolor{red}{\textbf{Solution for Problem 1:}}

Let M be the DFA with the following transition
diagram:\\
%% Machine generated by https://finsm.io
%% 2020-3-1-12:01:052
%% include in preamble:
%% \usepackage{tikz}
%% \usetikzlibrary{automata,positioning,arrows}
\begin{center}
\begin{tikzpicture}[]
    \node[line width = 0.55mm,state] at (-8.5,2.75) (02fb2fc7) {$q_{0}$};
    \node[thick,state] at (-6.75,2.75) (1ca2287e) {$q_{1}$};
    \node[thick,accepting,state] at (-4.75,2.75) (780aa57a) {$q_{2}$};
    \node[thick,state] at (-2.5,2.75) (f8690ee8) {$q_{3}$};
    \node[thick,accepting,state] at (0,2.75) (9a2a1c98) {$q_{4}$};
    \node[thick,accepting,state] at (2,2.75) (9b48d18d) {$q_{5}$};
    \node[thick,accepting,state] at (4.25,2.75) (065b6ae8) {$q_{6}$};
    \node[thick,state] at (4.25,0.25) (9a534d1e) {$q_{7}$};
    \node[thick,accepting,state] at (0.5,0.25) (d92ae927) {$q_{8}$};
    \node[thick,state] at (-3,0.25) (f004b18a) {$q_{9}$};
    \node[thick,accepting,state] at (-6.75,0.25) (9cf4bb96) {$q_{10}$};
    \path[->, thick, >=stealth]
    (02fb2fc7) edge [below,in = 180, out = 0] node {$a$} (1ca2287e)
    (1ca2287e) edge [below,in = 180, out = 0] node {$a$} (780aa57a)
    (780aa57a) edge [below,in = 180, out = 0] node {$a$} (f8690ee8)
    (f8690ee8) edge [below,in = 180, out = 0] node {$a$} (9a2a1c98)
    (9a2a1c98) edge [below,in = 180, out = 0] node {$a$} (9b48d18d)
    (9b48d18d) edge [below,in = 180, out = 0] node {$a$} (065b6ae8)
    (065b6ae8) edge [left,in = 90, out = -90] node {$a$} (9a534d1e)
    (9a534d1e) edge [above,in = 0, out = 180] node {$a$} (d92ae927)
    (d92ae927) edge [above,in = 0, out = 180] node {$a$} (f004b18a)
    (f004b18a) edge [above,in = 0, out = 180] node {$a$} (9cf4bb96)
    (9cf4bb96) edge [right,in = -90, out = 90] node {$a$} (1ca2287e)
    ;
\end{tikzpicture}
\end{center}

  \bigskip

  \item \textbf{[10 points]} Construct a deterministic finite
    automaton $M$ for the alphabet $\Sigma = \set{0,1}$ such that
    $L(M)$ is the set of all strings $x$ in $\Sigma^*$ for which
    $\#0(x)$ is divisible by 2 and $\#1(x)$ is divisible by 3.
    Present $M$ as a transition diagram.

  \bigskip

  \textcolor{blue}{\textbf{Name: Hishmat Salehi\\MacID: salehh6\\Date: \today}}

  \textcolor{red}{\textbf{Solution for Problem 2:}}

Let M be the DFA with the following transition
diagram:\\
%% Machine generated by https://finsm.io
%% 2020-3-1-20:57:8
%% include in preamble:
%% \usepackage{tikz}
%% \usetikzlibrary{automata,positioning,arrows}
\begin{center}
\begin{tikzpicture}[]
    \node[line width = 0.55mm,accepting,state] at (-0.8,-0.4) (f2c3d355) {$q_6$};
    \node[thick,state] at (-3.05,1.475) (99aa3f14) {$q_{1}$};
    \node[thick,state] at (-0.75,3.35) (18582618) {$q_{2}$};
    \node[thick,state] at (-8.95,-0.525) (031873a9) {$q_{3}$};
    \node[thick,state] at (-6.55,1.45) (facbb119) {$q_{4}$};
    \node[thick,state] at (-8.825,3.25) (9473005f) {$q_{5}$};
    \path[->, thick, >=stealth]
    (f2c3d355) edge [right,in = -40, out = 140] node {$1$} (99aa3f14)
    (f2c3d355) edge [above,in = 12, out = 170] node {$0$} (031873a9)
    (99aa3f14) edge [right,in = -141, out = 39] node {$1$} (18582618)
    (99aa3f14) edge [above,in = 24, out = 157] node {$0$} (facbb119)
    (18582618) edge [above,in = 12, out = 168] node {$0$} (9473005f)
    (18582618) edge [left,in = 89, out = -91] node {$1$} (f2c3d355)
    (031873a9) edge [right,in = -141, out = 39] node {$1$} (facbb119)
    (031873a9) edge [below,in = -172, out = -6] node {$0$} (f2c3d355)
    (facbb119) edge [right,in = -38, out = 142] node {$1$} (9473005f)
    (facbb119) edge [below,in = -156, out = -23] node {$0$} (99aa3f14)
    (9473005f) edge [below,in = -172, out = -7] node {$0$} (18582618)
    (9473005f) edge [left,in = 88, out = -92] node {$1$} (031873a9)
    ;
\end{tikzpicture}
\end{center}

\ee

\end{document}


